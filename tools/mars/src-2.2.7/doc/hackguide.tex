\documentclass{article}
\usepackage{url}
\usepackage[all]{xy}
\usepackage{graphicx}

\newcommand{\filename}[1]{{\tt #1}}
\newcommand{\cmdline}[1]{{\tt #1}}
\newcommand{\identifier}[1]{{\tt #1}}
\newcommand{\guiitem}[1]{{\sf #1}}
\newcommand{\xelement}[1]{{\tt <#1>}} 
\newcommand{\archc}[1]{*+<12pt>[F-:<12pt>]{\txt{#1}}}
\newcommand{\archto}[2]{\ar@/^/[#1]^{\txt{\scriptsize #2}}}
\newcommand{\openclasstable}{\begin{description}}
\newcommand{\closeclasstable}{\end{description}}
\newcommand{\classdesc}[2]{ \item[\identifier{#1}]{#2} }

\title{Mars 2.2.6 Hacker's Guide}
\author{Brian Trammell\\
\url{brian@lfrd.com}\\
{\small \sl Leapfrog Research and Development, LLC}}
\date{March 1, 2004}

\begin{document}
\sloppy
\maketitle

\begin{abstract}
Describes technical aspects of the Mars network status monitor, and provides a
guide for programmers wishing to extend Mars through the probe extension and
plugin frameworks.
\end{abstract}

\tableofcontents

\section{The Mars Configuration File}
\label{sec_marsconfig}
\begin{center}
\begin{figure*}
\begin{verbatim}
<mars:model xmlns:mars="http://www.altara.org/mars/xmlns/model/">  
  <mars:hostlist>  
    <mars:host
        name="Display name"
        address="DNS name or IP address">
      <mars:service
          name="Display name"
          port="TCP/UDP port"
          svctype="http|smtp|ftp|imap|pop3|ssh|tcp-connect"
          timeout="timeout in milliseconds"
          period="minimum wait time in milliseconds"
          notac="number of attempts before notification, 0-3">
        <mars:parameter name="name">value</mars:parameter>
        ... additional parameters, if any ...
        ... currently, only svctype http has a parameter, url, 
        ... for the path to check on the web server ...
      </mars:service>
      ... additional services ...
    </mars:host>
    ... additional hosts ...
  </mars:hostlist>
  ... plugin config elements, currently: ...
  <mars:swingnotify
      enabled="true or false"
      notifyBackUp="notify service back up, true or false"
      beep="audible notification, true or false"/>
  <mars:mailnotify
      enabled="true or false"
      notifyBackUp="notify service back up, true or false"
      server="DNS name or IP address of SMTP server to use"
      address="List of comma-separated recipient addresses"
      fromAddress="Origin address (optional)"/>
   <mars:csvlog
      enabled="true or false"
      logfile="path to logfile"/>
</mars:model>
\end{verbatim}
\caption{Mars configuration file structure}
\label{fig_marsconfig}
\end{figure*}
\end{center}

Though Mars was designed to be quick and easy to set up, using the Swing user
interface to configure many services on many hosts can get tedious, especially
if each host has many similar services.  In this case, creating a Mars
configuration file directly can be useful. 

The configuration file format fairly closely follows the data model as visible
from the host/service tree - a hostlist contains hosts, which each contain
services. The format of the file is given in figure \ref{fig_marsconfig}.

Mars itself is a good generator of example files; just create a simple
configuration, then open up the resulting XML file in your favorite text
editor.

\section{The Probe Definition File}
\label{sec_marsdef}
\begin{center}
\begin{figure*}
\begin{verbatim}
<mdef:definition xmlns:mdef="http://www.altara.org/mars/xmlns/def/">
  <mdef:svctype
      name="service type name"
      defaultPort="default TCP port">
    <mdef:param
        name="parameter name"
        label="parameter display label"
        default="default value or default variable"/>
    ... additional parameters, if any ...
    <mdef:script>
      ... see text for script syntax ...
    </mdef:script>
  </mdef:svctype>
  ... additional service types ...
</mdef:definition>
\end{verbatim}
\caption{Mars probe definition file structure}
\label{fig_probedef}
\end{figure*}
\end{center}	

Simple text-based TCP protocol probes are defined in the
\filename{mars-def.xml} file.  On startup, Mars searches for this file
in the Mars home directory, which is the current working directory by
default, though this can be changed using the \cmdline{--home}
command-line option. If a \filename{mars-def.xml} file cannot be found
in the Mars home directory, Mars falls back to a default version of
the file stored in the \filename{mars.jar} archive itself.

This file's format is a little more complex than the config file's.  Broadly
speaking, the definition file lists a collection of service types, each of
which has a list of parameters and a script that defines the protocol used by
the service. The structure is shown in figure \ref{fig_probedef}.

Each parameter has a name, by which it will be referenced in a
\xelement{mdef:param} element within the script; a label, which will be shown
to the user in the service editor; and a default, which is used as the
parameter's value if the user leaves it blank. The default may be any constant
string, or a default variable. The only currently supported default variable is
\identifier{\%\%(remote-hostname)}, which resolves to the DNS name or IP
address of the host on which the service is running.

Scripts are series of ``send'', ``expect'', and ``fail''
instructions, contained in \xelement{mdef:send}, \xelement{mdef:expect}, and
\xelement{mdef:fail} elements, respectively. If a dialogue with a server
matches all of a script's expect instructions in order without matching any
fail instructions, the probe succeeds.  The probe may time out waiting for an
expect. If a fail instruction matches, the probe fails with a status of ``bad
reply''. Fail instructions must be immediately followed by an expect
instruction, as fail instructions ``expire'' when the expect instruction
immediately following them successfully matches.

Each send instruction contains mixed content. The following elements are valid
within send instructions:

\begin{description}
\item[\xelement{mdef:param name}] expands to the value of the named parameter
as set for each service instance.
\item[\xelement{mdef:remote-hostname}] expands to the address of the host being
probed.
\item[\xelement{mdef:local-hostname}] expands to the hostname of the local
interface from which the probe is being run.
\item[\xelement{mdef:version}] expands to the Mars version number. This is
useful for signaling Mars' version to protocols that care about that sort of
thing; it is used to accurately report Mars' version on HTTP probes via the
\identifier{User-agent} request header.
\item[\xelement{mdef:space}] encodes a single space. This is useful for adding
leading or trailing spaces to mixed string content, as whitespace can be
otherwise normalized away.
\item[\xelement{mdef:crlf}] likewise encodes a newline sequence.
\end{description}

Each expect or fail instruction contains either a literal text string
or an \xelement{mdef:regex} element. These \identifier{regex}
elements match the Perl-compatible regular expression defined in the
\identifier{pattern} attribute, or filled in from the service
parameter named in the \identifier{name} attribute

As with the config file, it's probably best to look over the
\filename{mars-def.xml} file itself to get the big picture.

\section{Building Mars}
The Mars build system, contained in the files \filename{build.xml},
\filename{buildext.xml}, and \filename{buildmac.xml}, is based on
Apache Ant 1.5. The following targets are the most commonly used in
building Mars.

\begin{description} 
\item[all] (the default target) compiles any (changed) source files in
  the \filename{org} and \filename{gnu} directories, stages any
  unstaged \filename{.jar} files in \filename{lib} into
  \filename{lib/staging}, and creates or updates the \filename{.jar}
  files in the distribution directory (\filename{mars-2.2.4}). It also
  builds any optional extensions in the \filename{extras} directory
  named by the \identifier{extras} property (see below), and places
  them in the \filename{extras} directory in \filename{mars-2.2.4}.
\item[dist] runs \identifier{veryclean} (see below), then
  \identifier{all}, then makes a \filename{mars-2.2.4.tar.gz} binary
  distribution. On Mac OS X, this target also builds a Mac OS X
  installer package image in \filename{mars-2.2.4/macpkg}, suitable
  for use with the Package Maker specification file
  \filename{extras/macosx/Mars.pmsp}.
\item[clean] removes the distribution and object directories.
\item[veryclean] additionally removes the \filename{lib/staging}
  directory, where jarfiles in the \filename{lib} directory are staged
  for inclusion into \filename{mars.jar} and
  \filename{mars-j14.jar}. Use this to force the
  \filename{lib/staging} directory to be rebuilt when you change
  \filename{.jar} files in the \filename{lib} directory. This almost
  never happens.
\end{description}

The \identifier{extras} property used by the build system (set on the
Ant command line with \cmdline{-Dextras=a,b,c,...}) is a
comma-separated list of extras to build during the \identifier{all}
target. The following values are supported:

\begin{description}
\item[all] Build all extras.
\item[https] Build the HTTPS probe extension. This extension requires
  JSSE to be installed on Java versions earlier than 1.4.
\item[jdbc] Build the JDBC probe extension.
\item[xmpp] Build the XMPP (Jabber) notification plugin.
\item[cdebug] Build the Client Debugger plugin.
\item[mailnotify] Build the SMTP notification plugin.
\item[swingnotify] Build the UI notification plugin.
\item[csvlog] Build the CSV logging plugin.
\end{description}

The Mac OS X Integration plugin is always built by default on Mac OS
X. Note that this implies that building Mars on Mac OS X requires Java
1.4.

\section{Building Probe Extensions}

Some services protocols' may not be expressible in the probe definition file,
especially non-text-based or non-TCP services. These probes may be implemented
as probe extensions to be dynamically loaded at started, much like plugins are.
To implement a probe extension:

\begin{enumerate}
\item Implement a subclass of \identifier{Probe}. You should only need
  to implement a single method, \identifier{doProbe()}, which will
  take information about the service to probe via the protected
  \identifier{service} instance variable, probe the specified service,
  and return a \identifier{Status} instance describing the status of
  the service. See \identifier{SendExpectProbe} and
  \identifier{SendExpectClient}, which implement the probe
  infrastructure used by the probe definition file mechanism, for an
  example of a probe.
\item You should instrument your probe for the Client Debugger. This
  will allow you to verify your probe's functionality and allow users
  to view details your probe's operation in real time. Instrument your
  probe as follows:
  \begin{enumerate}
    \item At the beginning of \identifier{doProbe()}, call static
      method \identifier{Debug.getDebugger()} to get a new
      \identifier{ClientDebugger} instance. This method takes a single
      argument, a string which will be displayed to the user to
      identify this debug session. Try to use a string which will
      allow the user to distinguish multiple instances of your probe
      running concurrently.
    \item Each time you send data, pass the data (or some textual abstract
      representation of it) to your \identifier{ClientDebugger}'s
      \identifier{send()} method.
    \item Each time you receive data, pass the data (or some textual abstract
      representation of it) to your \identifier{ClientDebugger}'s
      \identifier{receive()} method.
    \item To notify of status not related to sending or receiving
      (for example, some decision your probe makes), use
      \identifier{ClientDebugger}'s \identifier{message()} method.
      \item When your probe completes, be sure to \identifier{close()}
        the \identifier{ClientDebugger}. This must be done or the
        debug session information will never be removed from the
        active sessions list. This call usually goes in a
        \identifier{finally} block within your probe.
  \end{enumerate}
\item Implement a subclass of \identifier{ProbeFactory} to create
  instances of your new probe, and to describe the parameters your
  probe requires to Mars. You will need to implement three methods:
  \begin{description}
  \item[A constructor] which calls \identifier{ProbeFactory()} with the
    name of the service type monitored by your probe. This service type
    name will be exposed to the user via the \guiitem{Type} menu in the
    service editor.
  \item[\identifier{createProbe(Service)}]creates a new probe instance
    to monitor the specified service.
  \item[\identifier{getDefaultPort()}] returns the default port used by
    the service.
  \end{description}
  If your probe takes any service parameters, you will also need to
  implement the following three methods:
  \begin{description}
  \item[\identifier{getServiceParamNames()}] returns an array of Strings
    containing the service parameter names required by this service
    type.
  \item[\identifier{getServiceParamLabels()}] returns an array of
    Strings containing the service parameter labels to be shown in the
    user interface, corresponding by position to the service parameter
    names.
  \item[\identifier{getServiceParamDefault(String)}] returns a String
    containing the default value for the named parameter.
  \end{description}
\item Place your probe and probe factory classes, along with any
  support classes they require, into a \filename{.jar} file. This
  file's name must begin with the string \filename{probe\_}, and its
  manifest must contain a key \identifier{MarsProbeFactoryClass} whose
  value names your \identifier{ProbeFactory} subclass.  You can place
  multiple probes into a single \filename{.jar}; simply name each
  \identifier{ProbeFactory} subclass in the
  \identifier{MarsProbeFactoryClass} separated by commas.
\item Place this file in the Mars distribution directory, and start
  Mars. Mars should detect your probe extension(s) on startup, and
  they should be available on the \guiitem{Type} menu in the service
  editor.
\end{enumerate}

\section{Building Plugins}

As probe extensions provide new inputs to Mars' monitoring engine, plugins
provide a way to add new outputs. Each plugin can be configured to receive
information about the results of each probe run, or about changes to each
service's status. 

Plugins receive status information via events recieved through three
interfaces, \identifier{ProbeListener},
\identifier{StatusChangeListener}, and
\identifier{NotificationListener}.  Plugin housekeeping tasks are
managed through a fourth interface, \identifier{Plugin}. Your plugin
class must implement \identifier{Plugin}, and at least one of the
three listener interfaces. Your class will not need to register as a
listener over these interfaces; the plugin framework handles this for
you as long as you have implement at least one of the interfaces.

The \identifier{ProbeListener} interface defines a single method,
\identifier{probeRun(ProbeEvent)}, which is called by the controller
every time a probe is run, even when the service status has
changed. The \identifier{ProbeEvent} instance passed to this method
contains information about the service that was probed via the
\identifier{getService()} method, and information about that service's
current status via the \identifier{getNewStatus()} method.

The \identifier{StatusChangeListener} interface also defines a single
method, \identifier{statusChanged(StatusChangeEvent)}, which is called
by the controller every time a service's status changes.
\identifier{StatusChangeEvent} is a subclass of
\identifier{ProbeEvent} supporting one additional method,
\identifier{getOldStatus()}, which returns the previous status of the
service.

The \identifier{NotificationListener} interface defines the
\identifier{notifyStatusChanged(StatusChangeEvent)} method, which is
called as specified by each service's notification retry count
(\identifier{notac}) field after a service's status changes. The
\identifier{StatusChangeEvent} instance is the same one that would
have been sent via \identifier{statusChanged()} when the service's
status initially changed.

Plugin configuration is done via editor panels displayed through the
\guiitem{Plugins} menu, and configuration data is stored
along with the host/service tree in the Mars configuration file. The
\identifier{Plugin} interface defines methods your plugin must implement to
support these configuration functions:

\begin{description}
\item[A constructor] taking no arguments, which should create a new instance of
your plugin with a default configuration. The plugins shipped with Mars
configure themselves with empty values assigned to each configuration parameter
and a global ``enabled'' flag set to off, to ensure that the user must
configure the plugin before its first run.
\item[\identifier{getConfig()}] returns a JDOM \identifier{Element} containing
all of the configuration information in an XML element suitable for inclusion
in the configuration file.
\item[\identifier{setConfig(Element)}] takes in a JDOM \identifier{Element},
previously returned by \identifier{getConfig()} and written to a configuration
file, and configures the plugin based on its attributes and contents.
\item[\identifier{getElementName()}] returns the configuration element name
within the Mars configuration XML namespace used by this plugin to identify its
configuration. 
\item[\identifier{getDisplayName()}] returns a string to be displayed in the
plugins list in the Mars main window's \guiitem{Config} tab.
\item[\identifier{getEditor()}] returns a \identifier{JComponent}
implementation of the \identifier{Editor} interface used to configure the
plugin via the user interface. Editors are often inner classes extending
\identifier{JPanel}. They must provide the following methods:
\begin{description}
\item[A constructor] that builds the editor panel's user interface and fills in
fields with values from the plugin's present configuration.
\item[\identifier{commit()}] configures the plugin based on the information in
the editor's editable fields. It may throw any \identifier{Exception} to
signify illegal input.
\item[\identifier{getEditorTitle()}] return a string to use as the editor
dialog's title.
\end{description}
\end{description}

As of Mars 2.2.6, plugins may now request a tab in the Mars main
window. This facility is designed for alternate visualization plugins.
To use this facility, your plugin must implement the
\identifier{Displayable} interface. \identifier{Displayable} defines a
single method, \identifier{setPluginDisplay()}, which Mars will call
when the user interface starts up (after configuration if a
configuration file is specified on the command line; see below). This
method takes a single argument, an instance of
\identifier{PluginDisplay} which your plugin will use to control its
tab in the user interface. This class has the following methods:

\begin{description}
\item[\identifier{setComponent()}] takes a \identifier{JComponent}
  instance to display in the plugin's tab.
\item[\identifier{setTitle()}] set's the plugin's tab's title.
\item[\identifier{show()}] makes your plugin's tab visible (call this
  when your plugin is enabled).
\item[\identifier{hide()}] makes your plugin's tab invisible (call this
  when your plugin is disabled).
\end{description}

You must set the plugin's tab's component and title before calling
\identifier{snow()}; otherwise, a runtime exception will result.  See
the source code of the Client Debugger in \filename{extras/cdebug} for
an example.

The \identifier{Displayable} interface is also used as a hint to Mars
that it should not attempt to load your plugin in ``headless''
(\identifier{\mbox{-}\mbox{-}nogui}) mode; if your plugin uses the
user interface but does not need its own tab (as the
\identifier{SwingNotifyPlugin} does), you can implement
\identifier{Displayable} but make your \identifier{setPluginDisplay()}
method a no-op.

Additionally, if your plugin class has an ``enabled'' state (as all
the included notification plugins do), you may implement the
\identifier{Enableable} interface to ensure that your plugin's enable
state is properly displayed in the \guiitem{Plugins}
menu. \identifier{Enableable} defines a single method,
\identifier{isEnabled()}, which returns \identifier{true} if the
plugin is presently enabled, \identifier{false} otherwise.

When a new Mars configuration file is loaded, each plugin is configured
according to the following protocol:

\begin{enumerate}
\item The plugin registry calls \identifier{getElementName()} on the
  plugin to determine the name of the XML element it uses for
  configuration.
\item If the configuration file contains that element, it passes that
  element to the plugin via \identifier{setConfig()}.
\item If no element of the given name exists in the configuration
  file, the plugin is {\em not} reconfigured. This allows older config
  files to be loaded with new plugins without resetting the plugin
  configuration.
\end{enumerate}

Note that the only time Mars calls your plugin's \identifier{setConfig()} is
when a configuration file is loaded that contains an element in the Mars
namespace with the element name returned by your \identifier{getElementName()}
method. In particular, this means that your plugin {\em will not be configured}
when Mars starts up without a configuration file. Your plugin's no-argument
constructor must leave the plugin in a state in which it will not crash or
cause other adverse effects for Mars.

Have a look at the source for \identifier{SwingNotifyPlugin} for an example of
a simple plugin.

To package your plugin for dynamic loading, place your plugin class
along with any support classes it requires into a \filename{.jar}
file. The file's name must begin with the string
\filename{plugin\_}. and its manifest must contain a key
\identifier{MarsPluginClass} which names your implementation of
\identifier{Plugin}. You can place multiple plugins into a single
\filename{.jar}; simply name each \identifier{Plugin} subclass in the
\identifier{MarsPluginClass} separated by commas.

Place this file in the Mars distribution directory, and start
Mars. Mars should detect your plugin(s) on startup, and you should be
able to configure them via the \guiitem{Plugins} menu.

\section{Architecture Overview}
\label{sec_archdes}
\begin{center}
\begin{figure*}
\begin{displaymath}
\xymatrix{
    \archc{Swing UI\\{\small \sl (view)}}
        \archto{rr}{edit}
        \archto{rdd}{control} & & 
    \archc{Host/Service Tree\\{\small \sl (model)}}
        \archto{ll}{display} \\
    & & \\
    \archc{Plugins} &
    \archc{Engine\\{\small \sl (controller)}}
        \archto{l}{notify}
        \archto{luu}{notify}
        \archto{ruu}{update}
        \archto{r}{run} &
    \archc{Probes}
        \archto{l}{status} 
}
\end{displaymath}
\caption{Mars architecture overview}
\label{fig_arch}
\end{figure*}
\end{center}

Mars is built along the lines of a straight MVC architecture. Information about
hosts and services lives in a Model (the \identifier{Host},
\identifier{Service}, and \identifier{Status} classes), which is edited and
displayed by a View (the Swing user interface in
\identifier{org.altara.mars.swingui} and the plugin extension framework), and
updated by the asynchronously-running Controller (the classes in
\identifier{org.altara.mars.engine}) which invokes probes (defined in the probe
definition file and the probe extension framework) to determine status
information. This general arrangement is shown in figure \ref{fig_arch}.

\section{Class reference}

The tables that follow list all the classes that make up Mars. This list
should suffice to get you started hacking away at Mars, by answering the the
frequent and essential questions ``where does feature {\sl x} live?'' and
``what does {\sl y} do?''. 

\subsection{Data Model}
Core data model classes in \identifier{org.altara.mars}:
\openclasstable
\classdesc{Main}{the entry point to Mars. It parses the command line,
initializes and manages access to plugins, probes, and the model, and contains
system-wide constants.}
\classdesc{MarsModel}{is the root of the host/service tree; it contains the
host list, and distributes host/service tree change notifications to the user
interface.}
\classdesc{Host}{represents a host Mars is monitoring.}
\classdesc{Service}{represents a service Mars is monitoring.}
\classdesc{Status}{represents the status of a service at a given point in time.}
\classdesc{ProbeFactory}{is an interface implemented by concrete factories that
can create probes for a given service type. It also contains a registry of such
factories, and handles dynamic loading of probe extensions at startup.}
\classdesc{MarsModelListener}{is an interface implemented by objects interested
in edits to the MarsModel tree.}
\classdesc{InvalidDocumentException}{is thrown during configuration file
loading if the file is not a proper Mars configuration file.}
\classdesc{InvalidServiceTypeException}{is thown during configuration file
loading if a service's type is not supported by any probe.}
\closeclasstable

\subsection{Monitoring Engine}
Monitoring engine classes in \identifier{org.altara.mars.engine}:
\openclasstable
\classdesc{Controller}{coordinates the activities of the engine. It distributes
event notifications, and manages the queue of pending probes and probe worker
threads.}
\classdesc{ProbeWorker}{runs Probes in separate threads.}
\classdesc{Probe}{is the abstract superclass of all probes; classes which know
how to speak a given protocol or set of protocols, and determine the status of
a remote service.}
\classdesc{SendExpectProbe}{is a thin wrapper around
\identifier{SendExpectClient} used to probe text-based TCP/IP services. This is
sufficient for all probes defined via the probe definition XML file.}
\classdesc{SendExpectClient}{is a simple text-based TCP/IP client that knows
how to send strings and expect strings in return. It is instrumented to return
a Status so it can be used as a probe; it's also used by the mail notification
plugin for a simple SMTP client.}
\classdesc{XmlProbeFactory}{parses the probe definition file and sets up a
\identifier{SendExpectProbe} for each service it finds.}
\classdesc{ProbeListener}{is implemented by any object interested in probe run
notifications.}
\classdesc{ProbeEvent}{is sent to each \identifier{ProbeListener} to notify of
a \identifier{Probe}'s being run.}
\classdesc{StatusChangeListener}{is implemented by any object interested in
service status change notifications.}
\classdesc{StatusChangeEvent}{is sent to each \identifier{StatusChangeListener}
to notify of a service status change.}
\classdesc{Notifier}{manages the firing of \identifier{NotificationEvents}.}
\classdesc{Debug}{contains static proxy methods for various debuggers}
\classdesc{ClientDebugger}{defines methods for network client debugging (these are implemented by the Client Debugger plugin)}
\closeclasstable

\subsection{User Interface}
User interface classes in \identifier{org.altara.mars.swingui}:
\openclasstable
\classdesc{MarsView}{implements the Mars main window.}
\classdesc{MarsAbstractRenderer}{provides common support to each Mars
  renderer class.}
\classdesc{ServiceTreeRenderer}{renders the nodes in the host/service tree.}
\classdesc{ServiceTreeChangeAdapter}{ensures that events received by
  the host/service tree run in the Swing dispatch thread.}
\classdesc{ServiceTreeContextMenuSupport}{implements the service tree
  context menu.}
\classdesc{ServiceTreeKeyActionSupport}{implements the service tree
  keyboard editing functions.}
\classdesc{FaultListModel}{implements a list model for the fault
  list. This is necessary to support fault hiding.}
\classdesc{FaultListRenderer}{renders the items in the fault list.}
\classdesc{FaultListContextMenuSupport}{implements the fault list
  context menu.}
\classdesc{DetailListModel}{implements the service details panel.}
\classdesc{ChangeListModel}{implements the change list in the
  \guiitem{History} tab.}
\classdesc{ChangeListRenderer}{renders the items in the change list.}
\classdesc{ChangeListPanel}{implements the \guiitem{History} panel.}
\classdesc{PluginMenu}{implements the \guiitem{Plugins} menu.}
\classdesc{Editor}{is implemented by Mars property editors: the host,
  service, and the plugin configuration editors.}
\classdesc{EditorDialog}{displays an Editor in a dialog and handles
  error reporting.}
\classdesc{HostEditorPanel}{implements the host editor.}
\classdesc{ServiceEditorPanel}{implements the service editor.}
\classdesc{ServiceTypeComboBox}{implements the \guiitem{Type} menu in
  the service editor.}
\classdesc{ServiceParamEditor}{implements the parameters pane in the
  service editor.}
\classdesc{ProbeThreadAdapter}{ensures that events sent to a
  \identifier{ProbeListener} are handled within the Swing dispatch
  thread.}
\classdesc{StatusChangeThreadAdapter}{ensures that events sent to a
  \identifier{StatusChangeListener} are handled within the Swing
  dispatch thread.}

\closeclasstable

\subsection{Plugins}
Plugins and support classes in \identifier{org.altara.mars.plugin}:
\openclasstable
\classdesc{Plugin}{is implemented by all plugins to provide plugin housekeeping
services.}
\classdesc{Enableable}{is implemented by plugins with an ``enabled'' state.}
\classdesc{Displayable}{is implemented by plugins that use the Swing user 
interface or display their own tab in the Mars main window.}
\classdesc{PluginDisplay}{is used by each \identifier{Displayable} plugin to 
control its tab in the Mars main window.}
\classdesc{PluginRegistry}{manages loading, instantiating, and configuring
plugins.}
\closeclasstable

\subsection{Utility classes}
Utility classes in \identifier{org.altara.util}:
\openclasstable
\classdesc{Queue}{implements a queue atop \identifier{LinkedList}.}
\classdesc{Worker}{provides a managable interface to a thread for
  running a repetitive task; it makes up for the unsafety of the
  \identifier{Thread} control methods.}
\classdesc{StatusView}{provides an interface for views with status
  bars or logs. It is used to implement the startup log in the Mars
  splash screen, and the Mars main window status bar.}
\classdesc{ExtensionLoader}{dynamically loads classes from
  \filename{.jar} files in a given directory.}
\classdesc{LoadExceptionHandler}{is implemented by classes that can
  handle exceptions during the extension loading process.}
\classdesc{IconService}{loads icons from a .jar file.}
\classdesc{ContextMenuSupport}{provides common support for context
  menus on Swing components.}
\classdesc{ListContextMenuSupport}{implements context menus on Swing
  \identifier{JList}s.}
\classdesc{TreeContextMenuSupport}{implements context menus on Swing
  \identifier{JTree}s.}
\classdesc{KeyActionSupport}{implements keyboard shortcuts on
  arbitrary Swing or AWT components.}
\classdesc{TreeKeyActionSupport}{implements keyboard shortcuts on
  Swing \identifier{JTree}s.}

\closeclasstable
\end{document}
