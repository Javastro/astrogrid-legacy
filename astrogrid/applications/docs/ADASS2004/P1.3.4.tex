\documentclass[11pt,twoside]{article}
\usepackage{adassconf}
\begin{document}   % Leave intact
\paperID{P1.3.4}

\title{The AstroGrid Common Execution Architecture (CEA)}



\author{Paul Harrison}
\affil{AstroGrid, Jodrell Bank Observatory, Manchester University, UK}
\author{Noel Winstanley}
\affil{AstroGrid, Jodrell Bank Observatory, Manchester University, UK}
\author{John Taylor}
\affil{AstroGrid, Royal Observatory Edinburgh, UK}
\contact{Paul Harrison}
\email{paul.harrison@manchester.ac.uk}
\paindex{Harrison, P.A.}
\aindex{Winstanley, N}
\aindex{Taylor, J. D.}
\authormark{Harrison, Winstanley \& Taylor}
\keywords{software: architecture, AstroGrid, Web: Services, Observatory:
virtual}

\begin{abstract}
The UK Virtual Observatory (VO) project AstroGrid (see \htmladdURL{http://www.astrogrid.org}
and related talks at this meeting) began in 2001 and is nearing the successful
completion of its first release in December 2004.

 This paper describes AstroGrid's Common Execution Architecture (CEA). This is
an attempt to create a reasonably small set of interfaces and schema to model
how to execute a typical astronomical application within the VO. 

The CEA has been designed primarily to work within a web services framework,
with the parameter passing mechanism layered on top of this so that the web
interface for all applications is described by a single constant piece of WSDL
- the differences between applications are expressed by the registry entries
for each application.  Within AstroGrid we have created pluggable components
that can wrap legacy command-line applications, HTTP GET/POST applications and
databases as CEA compliant web services, which when combined with the Astrogrid
Workflow component make distributed processing within the VO a reality.
See \htmladdURL{http://www.astrogrid.org/maven/docs/snapshot/applications/}
for current information.
\end{abstract}
\section{Introduction}


 The Common Execution Architecture (CEA) is an attempt to create a reasonably
 small set of interfaces and schema to model how to execute a typical
 Astronomical application within the Virtual Observatory (VO). In this context
 an application can be any process that consumes or produces data, so in
 existing terminology could include;
\begin{itemize}
\item 

 A Unix command line application 

\item 

 A database query 

\item 

 A web service 


\end{itemize}


 The CEA has been primarily designed to work within a web services calling
 mechanism, although it is possible to have specific language bindings using
 the same interfaces. For example Astrogrid has a Java implementation of the
 interfaces that can be called directly from a Java executable.
\subsection{Motivation}


 The primary requirements motivating the creation of this architecture are;
\begin{itemize}
\item 

 To create a uniform interface and model for an application and its parameters. This has twin benefits; 



\begin{enumerate}
\item 

 It allows VO infrastructure writers a single model of an application that that have to code for. 

\item 

 Application writers know what they have to implement to be compatible with a VO Infrastructure. 


\end{enumerate}

\item 

 To provide a higher level description than WSDL 1.1 can offer. 

\item 

 Restrict the almost limitless possibilities allowed by WSDL into a manageable subset. 

\item 

 Provide specific semantics for some astronomical quantities. 

\item 

 Provide extra information not allowed in WSDL - e.g. default values, descriptions for use in a GUI etc. 

\item 

 To provide extensions with the VO Resource schema (See the IVOA WG) that can describe a general application 

\item 

 To provide asynchronous operation of an application - This is essential as the
 call tree that invokes the application cannot be expected to be active for
 extremely long lasting operations - e.g. a user from a web browser invokes a
 data-mining operation that takes days

\item 

 Provide callback for notification of finishing. 

\item 

 Provide polling mechanisms for status.

\item 

 To allow for the data flow to not necessarily have to follow the call tree. In
 a typical application execution the results are returned to the invoking
 process - In a VO scenario, it can be useful if the application can be
 instructed to pass the results on to a different location


\end{itemize}
\subsection{Origins}

 Amongst the VO specifications there was no existing model for applications
 that was defined at the level at which this design attempts to address. In the
 VOResource schema an application is defined as a Service with the interface
 definition. The interface definition either relies on referring to a WSDL
 definition of the service, or on other schema extending the service definition
 to provide some specific detail as in the case of a Simple Image Access
 service. There is no general definition of an application in the resource specification.


 It is clear that the WSDL model of an interface has had a large influence on
 the design of the CEA, but it should be remembered that the CEA is
 intentionally layered on top of WSDL,� so that CEA controls the scope and
 semantics of operations. There is only one WSDL definition for all
 applications, so as far as web services are concerned the interface is
 constant.� CEA works by transporting meta information about the application
 interface within this constant WSDL interface.
\section{Interfaces}
\begin{figure}
\plotone{P1.3.4_1.epsi}
\caption{CEA Call Sequence}
\end{figure}
 The 3 WSDL ports that are used to interact within CEA are briefly described below;
\subsubsection{CommonExecutionConnector}
 This is the main port that is used to communicate with the application. The main operations in this port are;
\begin{itemize}
\item 
 \textbf{init}
 - this will initialize the application environment - returns and executionId
by which any particular execution run can be referenced later. 
\item 
 \textbf{registerResultsListener}
 - any number of services can register themselves as wanting to receive the results from the run when they are available as long as they implement the ResultsListener port below 
\item 
 \textbf{registerProgressListener}
 - any number of services can register themselves as wanting to receive status messages during the run as long as they implement the JobMonitor port below 

\item 

 \textbf{execute}
 - will actually start the asynchronous execution of the application specified in the init call. 

\item 

 \textbf{queryExecutionStatus}
 - this call can be used to actively obtain the execution status of a running application, rather than passively waiting for it as a JobMonitor 

\item 

 \textbf{abort}
 - will attempt to abort the execution of an application 

\item 

 \textbf{getExecutionSummary}
 - request summary information about the application execution 

\item 

 \textbf{getResults} - actively request the results of the application
 execution, rather than passively waiting for them as a ResultsListener.

\item 

 \textbf{returnRegistryEntry} - this returns the registry entry for the
 particular CommonExecutionConnector instance - this will probably be removed
 from this interface to be replaced by the equivalent operation in the standard
 VO service definitions.
\end{itemize}
\subsubsection{JobMonitor}
 The only operation in the JobmMonitor port is the monitorJob operation, which
 expects to receive a message with the job-identifier-type (as specified in the
 original init operation of the CommonExectutionConnector port) and a status
 message
\subsubsection{ResultsListener}
 The only operation is the putResults on the ResultsListener port. This accepts
 a message that contains a job-identifier-type and a result-list-type, which is
 just a list of parameterValues.
\subsection{Objects}  

 The objects that participate in CEA can be split into two groups
\begin{enumerate}
\item 

 Those used to describe the application in the registry 
\begin{itemize}
\item 

 \emph{Application}
 the overall application, which has a series of 

\item 

 \emph{Parameter }
which are the detailed descriptions of the parameters and their types. 


\end{itemize}

\item 

 Those used to describe the application in the WSDL interface 
\begin{itemize}
\item 

 \emph{Tool}
 - An instance of an application with real parameter values. 

\item 

 \emph{ParameterValue}
 which is used to pass a values to a Tool. 


\end{itemize}


\end{enumerate}
\subsubsection{Application}


 As this model depicts an application in CEA is really quite a simple entity
 consisting of 1 or more interfaces that consist of 0 or more input parameters
 and 0 or more output parameters.
\subsubsection{Parameter Definition}


 The description of the parameters and the parameter values are probably the
 heart of the CEA. It is the model for the parameters that allow us to add
 semantic meaning, and to give the flexibility in how the parameters are
 transported. The implementation is still in its infancy, but it is hoped that
 the parameter definition will be extended to encompass any data models that
 the VO produces.
\subsubsection{Tool}


 The tool represents the full collection of parameters that are passed to a
 particular interface of an application and the results that are returned.
\subsubsection{ParameterValue}


 The parameterValue model is simple but powerful representation of the
 parameters that are passed to an application. The parameterValue element has a
Value subelement with a string representation of the value as well as having  2
 attributes
\begin{itemize}
\item 

 \textbf{name} - The name of the parameter. 

\item 

 \textbf{indirect} - This describes whether the value element of the parameter should be
 used as is (indirect=''false''), or if the value of the parameter
 represents a URI from which the actual value should be fetched
 (indirect=''true''). It has not been defined what is the minimum set of transport
 mechanisms a service should understand to be CEA compliant, but the different
 sorts of transport mechanism are expected to include
\begin{itemize}
\item 

 SOAP messages 

\item 

 http get/put 

\item 

 SOAP attachments 

\item 

 ftp/gridftp 

\item 

 MySpace 

\item 

 local filestore 

\end{itemize}


\end{itemize}
\section{Astrogrid Implementation}
\subsection{CEA}


 AstroGrid has created a set of 
 \htmladdnormallink{software that implements the CEA}{http://www.astrogrid.org/maven/docs/snapshot/applications/}. The main
 server components that implement the CommonExecutionConnector web service are
 called ExecutionControllers and there are currently two specialized adapter
 controllers for
\begin{itemize}
\item 

 Wrapping legacy command line applications in the framework.

\item 

 Wrapping legacy GET/POST style HTTP applications in the framework.


\end{itemize}

 There are plans for a further adapter for other web services themselves as
 well as the possibility of creating new applications that directly implement
 the CEA.
\subsection{Job Execution System (JES)/Workflow}

 The CEA only specifies how to call applications, there is another component,  \htmladdnormallinkfoot{JES}{http://www.astrogrid.org/maven/docs/SNAPSHOT/jes/},
 with the responsibility for actually organising the order of execution of a
 sequence of applications.
. The
 job system in AstroGrid is driven from the workflow which has an
 \htmladdnormallink{xml schema
 representation}{http://www.astrogrid.org/viewcvs/*checkout*/astrogrid/workflow-objects/schema/Workflow.xsd?rev=HEAD&content-type=text/plain}.
 The unit of execution within the workflow is a Step, which can itself be part
 of a Sequence (a set of actions performed in series) or a Flow (a set of
 actions performed in parallel). The Step contains a Tool which is the full CEA
 description of a particular execution instance of an application. This paper
 does not discuss the full complexity of the workflow which also includes a
 scripting component known as JEScript (based on the Java scripting language
 Groovy) which allows workflows to be arbitrarily complex.
\end{document}
